\documentclass[portrait,final,a0paper]{baposter}
%\documentclass[a4shrink,portrait,final]{baposter}
% Usa a4shrink for an a4 sized paper.

\tracingstats=2

\usepackage{calc}
\usepackage{graphicx}
\usepackage{amsmath}
\usepackage{amssymb}
\usepackage{relsize}
\usepackage{multirow}
\usepackage{bm}

\usepackage{multicol}

\usepackage{pgfbaselayers}
\pgfdeclarelayer{background}
\pgfdeclarelayer{foreground}
\pgfsetlayers{background,main,foreground}

\usepackage{times}
\usepackage{helvet}
%\usepackage{bookman}
%\usepackage{palatino}

\newcommand{\captionfont}{\footnotesize}


\graphicspath{{images/}}

%%%%%%%%%%%%%%%%%%%%%%%%%%%%%%%%%%%%%%%%%%%%%%%%%%%%%%%%%%%%%%%%%%%%%%%%%%%%%%%%
% Multicol Settings
%%%%%%%%%%%%%%%%%%%%%%%%%%%%%%%%%%%%%%%%%%%%%%%%%%%%%%%%%%%%%%%%%%%%%%%%%%%%%%%%
\setlength{\columnsep}{0.7em}
\setlength{\columnseprule}{0mm}


%%%%%%%%%%%%%%%%%%%%%%%%%%%%%%%%%%%%%%%%%%%%%%%%%%%%%%%%%%%%%%%%%%%%%%%%%%%%%%%%
% Save space in lists. Use this after the opening of the list
%%%%%%%%%%%%%%%%%%%%%%%%%%%%%%%%%%%%%%%%%%%%%%%%%%%%%%%%%%%%%%%%%%%%%%%%%%%%%%%%
\newcommand{\compresslist}{%
\setlength{\itemsep}{1pt}%
\setlength{\parskip}{0pt}%
\setlength{\parsep}{0pt}%
}


%%%%%%%%%%%%%%%%%%%%%%%%%%%%%%%%%%%%%%%%%%%%%%%%%%%%%%%%%%%%%%%%%%%%%%%%%%%%%%
%%% Begin of Document
%%%%%%%%%%%%%%%%%%%%%%%%%%%%%%%%%%%%%%%%%%%%%%%%%%%%%%%%%%%%%%%%%%%%%%%%%%%%%%
\newcommand{\ShowFig}[3]{%
\begin{center}
 \includegraphics[height=0.15\textheight]{#1.png}
\\
\refstepcounter{figure}\label{#3}%
{Figure \arabic{figure}: #2}
\end{center}
}
\newcommand{\ShowFigS}[4]{%
\def\SSS{#2}
  \begin{center}
   \displayFigs#1\relax
\\
\refstepcounter{figure}\label{#4}%
{Figure \arabic{figure}: #3}
  \end{center}
}
\newcommand{\displayFigs}[1]{
\ifx#1\relax\relax\else
 \includegraphics[height=\SSS\textheight]{#1.png}\expandafter\displayFigs
 \fi
} 
\begin{document}

%%%%%%%%%%%%%%%%%%%%%%%%%%%%%%%%%%%%%%%%%%%%%%%%%%%%%%%%%%%%%%%%%%%%%%%%%%%%%%
%%% Here starts the poster
%%%---------------------------------------------------------------------------
%%% Format it to your taste with the options
%%%%%%%%%%%%%%%%%%%%%%%%%%%%%%%%%%%%%%%%%%%%%%%%%%%%%%%%%%%%%%%%%%%%%%%%%%%%%%
% Define some colors
\iftrue
\selectcolormodel{cmyk}

\definecolor{lightgray}{cmyk}{0,0,0,0.1}
\definecolor{yellow}{cmyk}{0,0,0.9,0.0}
\definecolor{reddishyellow}{cmyk}{0,0.22,1.0,0.0}
\definecolor{black}{cmyk}{0,0,0.0,1.0}
\definecolor{darkYellow}{cmyk}{0,0,1.0,0.5}
\definecolor{darkSilver}{cmyk}{0,0,0,0.1}

\definecolor{lightyellow}{cmyk}{0,0,0.3,0.0}
\definecolor{lighteryellow}{cmyk}{0,0,0.1,0.0}
\definecolor{lighteryellow}{cmyk}{0,0,0.1,0.0}
\definecolor{lightestyellow}{cmyk}{0,0,0.05,0.0}
\else
\selectcolormodel{rgb}

\definecolor{lightgray}{rgb}{0.95,0.95,0.9}%%
\definecolor{yellow}{rgb}{1,1,0.0}%%
\definecolor{reddishyellow}{rgb}{1,0.9,0}
\definecolor{black}{rgb}{0,0,0.0}%%
\definecolor{darkYellow}{rgb}{0.8,0.8,0}
\definecolor{darkSilver}{rgb}{0,0,0}

\definecolor{lightyellow}{rgb}{1.000,1.000,0.878}%%
\definecolor{lighteryellow}{rgb}{1,1,0.9}
\definecolor{lightestyellow}{rgb}{1,1,0.95}
\fi
%%
\typeout{Poster Starts}
\iffalse\else
\background{
  \begin{tikzpicture}[remember picture,overlay]%
    \draw (current page.north west)+(-2em,2em) node[anchor=north west] {\includegraphics[height=1.1\textheight]{silhouettes_background}};
  \end{tikzpicture}%
}
\fi
\newlength{\leftimgwidth}
\begin{poster}%
  % Poster Options
  {
  % Show grid to help with alignment
  grid=no,
  columns=4,
  % Column spacing
  colspacing=1em,
  % Color style
  bgColorOne=lightgray,
  bgColorTwo=lighteryellow,
  borderColor=reddishyellow,
  headerColorOne=yellow,
  headerColorTwo=reddishyellow,
  headerFontColor=black,
  boxColorOne=lightyellow,
  boxColorTwo=lighteryellow,
  % Format of textbox
  textborder=roundedleft,
%  textborder=rectangle,
  % Format of text header
  eyecatcher=no,
  headerborder=open,
  headerheight=0.115\textheight,
  headershape=roundedright,
  headershade=shade-lr,%plain,
  headerfont=\Large\textsf, %Sans Serif
  boxshade=shade-lr,%plain,
%
  background=shade-tb,
%  background=plain,
  linewidth=2pt
  }
  % Eye Catcher
  {%\includegraphics[width=10em]{D1077}% No eye catcher for this
 %poster. (eyecatcher=no above). If an eye catcher is present, the title
 %is centered between eye-catcher and logo.
 }
  % Title
  {%\sf %Sans Serif
  %
\bf% Serif
 \vspace*{2em}
 \Huge
 \hspace*{1em}\protect\scalebox{1.5}{Increase amount of brightness}\\
 \hspace*{2em}\protect\scalebox{1.5}{in comparison of brightness}
\vspace{-0.6em}
 }
  % Authors
  {\sf %Sans Serif
  % Serif
\begin{center}
\begin{minipage}[b][0em]{5em}
\raisebox{-3\baselineskip}%
%{\includegraphics[width=\textwidth]{./images/ouchi-anag.png}}%
{}\end{minipage}
\begin{minipage}[t]{0.8\textwidth}
\begin{center}
\hspace*{2em}Teluhiko Hilano
(Kanagawa Institute of Technology)\\
 %\hspace*{2.5em}
 hilano@ic.kanagawa-it.ac.jp
\end{center}
\end{minipage}
\end{center}
  }
  % University logo
  {% The makebox allows the title to flow into the logo, this is a hack
 % because of the L shaped logo.
\includegraphics[width=10em]{./kait-logo.png}
    }
  
%%%%%%%%%%%%%%%%%%%%%%%%%%%%%%%%%%%%%%%%%%%%%%%%%%%%%%%%%%%%%%%%%%%%%%%%%%%%%%
%%% Now define the boxes that make up the poster
%%%---------------------------------------------------------------------------
%%% Each box has a name and can be placed absolutely or relatively.
%%% The only inconvenience is that you can only specify a relative position 
%%% towards an already declared box. So if you have a box attached to the 
%%% bottom, one to the top and a third one which should be in between, you 
%%% have to specify the top and bottom boxes before you specify the middle 
%%% box.
%%%%%%%%%%%%%%%%%%%%%%%%%%%%%%%%%%%%%%%%%%%%%%%%%%%%%%%%%%%%%%%%%%%%%%%%%%%%%%
    %
 \headerbox{An Effect of Contrast}{name=contrast,column=0,row=0,span=2}{%
The appearance of a color depends on the colors near it. In two squares
 with center circles of the same color , the brightness of the circles
 will appear to differ depending on the brightness of their
 backgrounds. 

 \ShowFigS{{fig-0-50}{fig-90-50}}{0.1}{An effect of contrast}{contrast}
 The brighter the background, the darker its circle
 appears.

 In this research, we want to quantify perception of the difference in the
 brightness of the circles.}
 \headerbox{The experimental way}
 {name=sample,column=0,below=contrast,span=2}
 {
 We prepared many printed samples with center
 circles of varying brightness. The size of the sample is 8 cm and the
 diameter of the circle is 5 cm. The background colors are black and
 lightgray(90\% brightness) and the brightness of the center circles are
 from 40\% to 60\% in every 2\% by HSL color value\cite{HSL}.
 These samples are printed by Epson PX5002 inkjet printer on the paper
 Epson Crispia.

\ShowFigS{{fig-0-50}}{0.1}{The refference Figure}{fig1}
\ShowFigS{{fig-90-52}{fig-90-54}{fig-90-56}}{0.1}{Example of
 Samples(gray of brightness 52\%, 54\% and 56\%)}{fig2}
The brightness of the backgrounds in Figures\ref{fig1} and \ref{fig2}
 are $0\%$ and $90\%$, respectively. 
 
 We showed these samples to subjects and
 asked them to select a sample whose center circle has the same center
 brightness as the reference image. }
  \headerbox{Measurement of Colors}
 {name=sample,column=0,below=sample,span=2}
 {Since the printed color depends on the equipmnts,
 we have measured the colors of athe samples by a colorimeter
 TCD100\cite{TCD100}. The colore differences are culculaed by Excel
 datasheet given by \cite{CIEDE2000}.
  \begin{center}
\refstepcounter{table}Table \arabic{table}: Color Values of the circles in
   Figure \ref{fig2}\\
 \begin{tabular}[t]{|*{7}{r|}}\hline
\multicolumn{1}{|c|}{$H$} &
\multicolumn{1}{c|}{$S$} &
\multicolumn{1}{c|}{$L$} &
\multicolumn{1}{|c|}{$L^*$} &
\multicolumn{1}{c|}{$a^*$} &
\multicolumn{1}{c|}{$b^*$} &
\multicolumn{1}{|c|}{$dE$} \\\hline
  $0$&$0$ &$50$ &$-$ &$-$ & $-$& $0$\\\hline
  $0$&$0$ &$52$ & & & & \\\hline
  $0$&$0$ &$54$ & & & & \\\hline
  $0$&$0$ &$56$ & & & & \\\hline
 \end{tabular}
\end{center}
}
  \headerbox{Results}
 {name=results,column=2,row=0,span=2}{}
  \headerbox{Conclusion}
 {name=conclusion,column=2,below=results,span=2}
{The colors of the samples were
 evaluated in the L*a*b* color space by a colorimeter. The background
 color and circle brightness in the reference image are 80 and 60,
 respectively. When the samples' brightness of the background color were
 40, approximately 6.6 differences were obtained. }
\headerbox{References}{name=references,column=2,span=2,above=bottom}{
    \smaller
    \vspace{-0.4em}
    \bibliographystyle{ieee}
    \renewcommand{\section}[2]{\vskip 0.05em}
      \begin{thebibliography}{1}\itemsep=-0.01em
      \setlength{\baselineskip}{0.4em}
       \bibitem{CIEDE2000}G. Sharma, W. Wu,E. N. Dalal,
               The CIEDE2000 Color-Difference Formula:
               Implementation Notes, Supplementary Test Data, and
               Mathematical Observations, COLOR research and
               application, pp.21-30, Vol.30, 2005
       \bibitem{HSL}W3C, CSS Color Module Level 3,
               https://www.w3.org/TR/css-color-3/\#hsl-color
       \bibitem{TCD100}TCD100,
       https://www.pce-instruments.com/f/english/media/colorimeter-catalog.pdf
       \bibitem{Ninio}Jacques Ninio, {\em The Science of Illusions}, Cornell
	       University Press, New York, 2001 
      \end{thebibliography}
  }
\end{poster}
\end{document}
